    \hypertarget{ux4ee5openmlux4e2dux7684ux6570ux636eux96c61studentperfromanceux4e3aux4f8bux6c42ux6570ux5b66ux5206ux6570math-scoreux8fdeux7eedux51faux73b03ux6b21ux7684ux5206ux6570}{%
\subsubsection{以openml中的数据集1StudentPerfromance为例,求数学分数math
score连续出现3次的分数。}\label{ux4ee5openmlux4e2dux7684ux6570ux636eux96c61studentperfromanceux4e3aux4f8bux6c42ux6570ux5b66ux5206ux6570math-scoreux8fdeux7eedux51faux73b03ux6b21ux7684ux5206ux6570}}

    \begin{tcolorbox}[breakable, size=fbox, boxrule=1pt, pad at break*=1mm,colback=cellbackground, colframe=cellborder]
\prompt{In}{incolor}{1}{\boxspacing}
\begin{Verbatim}[commandchars=\\\{\}]
\PY{c+c1}{\PYZsh{} 导入数据集获取工具}
\PY{k+kn}{from} \PY{n+nn}{sklearn}\PY{n+nn}{.}\PY{n+nn}{datasets} \PY{k+kn}{import} \PY{n}{fetch\PYZus{}openml}
\PY{c+c1}{\PYZsh{} 导入数据分析库}
\PY{k+kn}{import} \PY{n+nn}{pandas} \PY{k}{as} \PY{n+nn}{pd}
\end{Verbatim}
\end{tcolorbox}

    \begin{tcolorbox}[breakable, size=fbox, boxrule=1pt, pad at break*=1mm,colback=cellbackground, colframe=cellborder]
\prompt{In}{incolor}{2}{\boxspacing}
\begin{Verbatim}[commandchars=\\\{\}]
\PY{c+c1}{\PYZsh{} 获取数据}
\PY{n}{data} \PY{o}{=} \PY{n}{fetch\PYZus{}openml}\PY{p}{(}
    \PY{n}{data\PYZus{}id}\PY{o}{=}\PY{l+m+mi}{43255}\PY{p}{,}
    \PY{n}{as\PYZus{}frame}\PY{o}{=}\PY{k+kc}{True}\PY{p}{,}
    \PY{n}{parser}\PY{o}{=}\PY{l+s+s2}{\PYZdq{}}\PY{l+s+s2}{pandas}\PY{l+s+s2}{\PYZdq{}}
\PY{p}{)}\PY{p}{[}\PY{l+s+s2}{\PYZdq{}}\PY{l+s+s2}{frame}\PY{l+s+s2}{\PYZdq{}}\PY{p}{]}
\end{Verbatim}
\end{tcolorbox}

    \begin{tcolorbox}[breakable, size=fbox, boxrule=1pt, pad at break*=1mm,colback=cellbackground, colframe=cellborder]
\prompt{In}{incolor}{3}{\boxspacing}
\begin{Verbatim}[commandchars=\\\{\}]
\PY{c+c1}{\PYZsh{} 查看数据集的前五行}
\PY{n}{data}\PY{o}{.}\PY{n}{head}\PY{p}{(}\PY{p}{)}
\end{Verbatim}
\end{tcolorbox}

            \begin{tcolorbox}[breakable, size=fbox, boxrule=.5pt, pad at break*=1mm, opacityfill=0]
\prompt{Out}{outcolor}{3}{\boxspacing}
\begin{Verbatim}[commandchars=\\\{\}]
   gender race/ethnicity parental level of education         lunch  \textbackslash{}
0  female        group B          bachelor\textbackslash{}'s degree      standard
1  female        group C                some college      standard
2  female        group B            master\textbackslash{}'s degree      standard
3    male        group A         associate\textbackslash{}'s degree  free/reduced
4    male        group C                some college      standard

  test preparation course  math score  reading score  writing score
0                    none          72             72             74
1               completed          69             90             88
2                    none          90             95             93
3                    none          47             57             44
4                    none          76             78             75
\end{Verbatim}
\end{tcolorbox}
        
    \begin{tcolorbox}[breakable, size=fbox, boxrule=1pt, pad at break*=1mm,colback=cellbackground, colframe=cellborder]
\prompt{In}{incolor}{4}{\boxspacing}
\begin{Verbatim}[commandchars=\\\{\}]
\PY{c+c1}{\PYZsh{} 一步差分结果,删除缺失值}
\PY{n}{diff1} \PY{o}{=} \PY{n}{data}\PY{p}{[}\PY{l+s+s2}{\PYZdq{}}\PY{l+s+s2}{math score}\PY{l+s+s2}{\PYZdq{}}\PY{p}{]}\PY{o}{.}\PY{n}{diff}\PY{p}{(}\PY{p}{)}\PY{o}{.}\PY{n}{dropna}\PY{p}{(}\PY{p}{)}
\PY{n}{diff1}
\end{Verbatim}
\end{tcolorbox}

            \begin{tcolorbox}[breakable, size=fbox, boxrule=.5pt, pad at break*=1mm, opacityfill=0]
\prompt{Out}{outcolor}{4}{\boxspacing}
\begin{Verbatim}[commandchars=\\\{\}]
1      -3.0
2      21.0
3     -43.0
4      29.0
5      -5.0
       {\ldots}
995    25.0
996   -26.0
997    -3.0
998     9.0
999     9.0
Name: math score, Length: 999, dtype: float64
\end{Verbatim}
\end{tcolorbox}
        
    \begin{tcolorbox}[breakable, size=fbox, boxrule=1pt, pad at break*=1mm,colback=cellbackground, colframe=cellborder]
\prompt{In}{incolor}{5}{\boxspacing}
\begin{Verbatim}[commandchars=\\\{\}]
\PY{c+c1}{\PYZsh{} 找到为零的那些元素,表示重复两次出现的值}
\PY{n}{diff1\PYZus{}zero} \PY{o}{=} \PY{n}{diff1}\PY{p}{[}\PY{n}{diff1}\PY{o}{==}\PY{l+m+mi}{0}\PY{p}{]}
\PY{n}{diff1\PYZus{}zero}
\end{Verbatim}
\end{tcolorbox}

            \begin{tcolorbox}[breakable, size=fbox, boxrule=.5pt, pad at break*=1mm, opacityfill=0]
\prompt{Out}{outcolor}{5}{\boxspacing}
\begin{Verbatim}[commandchars=\\\{\}]
384    0.0
390    0.0
432    0.0
437    0.0
453    0.0
518    0.0
537    0.0
550    0.0
565    0.0
747    0.0
797    0.0
925    0.0
948    0.0
Name: math score, dtype: float64
\end{Verbatim}
\end{tcolorbox}
        
    \begin{tcolorbox}[breakable, size=fbox, boxrule=1pt, pad at break*=1mm,colback=cellbackground, colframe=cellborder]
\prompt{In}{incolor}{6}{\boxspacing}
\begin{Verbatim}[commandchars=\\\{\}]
\PY{c+c1}{\PYZsh{} 再对一步差分结果中零的索引进行差分,若为零,则说明是三个连续值}
\PY{n}{result} \PY{o}{=} \PY{n}{pd}\PY{o}{.}\PY{n}{Series}\PY{p}{(}\PY{n}{diff1\PYZus{}zero}\PY{o}{.}\PY{n}{index}\PY{p}{)}\PY{o}{.}\PY{n}{diff}\PY{p}{(}\PY{p}{)}\PY{o}{.}\PY{n}{dropna}\PY{p}{(}\PY{p}{)}
\PY{n}{result}\PY{p}{[}\PY{n}{result}\PY{o}{==}\PY{l+m+mi}{0}\PY{p}{]}
\end{Verbatim}
\end{tcolorbox}

            \begin{tcolorbox}[breakable, size=fbox, boxrule=.5pt, pad at break*=1mm, opacityfill=0]
\prompt{Out}{outcolor}{6}{\boxspacing}
\begin{Verbatim}[commandchars=\\\{\}]
Series([], dtype: float64)
\end{Verbatim}
\end{tcolorbox}
